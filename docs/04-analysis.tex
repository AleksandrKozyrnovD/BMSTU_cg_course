\chapter{Аналитический раздел}

В данном разделе проводится выбор и анализ существующих алгоритмов построения изображения
трехмерной сцены.


\section{Описание объектов сцены}

Сцена состоит из набора следующих объектов:
\begin{itemize}
    \item Площадка --- правильный параллелипипед, состоящий из сетки ячеек, на
        которой размещаются объекты сцены. Размеры площадки задаются по ширине и 
        длине одновременно. Ячейка --- заранее определенный объект, определенный внутри
        программы;

    \item Объект сцены --- модель, расположенная на ячейке сцены. Данные объекты
        могут занимать только одну ячейку сцены одновременно. Каждая такая модель
        описана массивом граней с нормалями. Все доступные объекты сцены
        определены заранее, и в программе не предусмотрена возможность изменения
        старых и добавления новых объектов сцены. Данные модели можно
        перемещать по сцене и вращать относительно их заданного центра.

    \item Источник света --- точка в пространстве. Источник света имеет координаты,
        направление и интенсивность освещения. В зависимости от расположения
        и направления источника света относительно камеры определяются тени от объектов
        сцены.

    \item Камера --- точка в пространстве. От ее положения и направления взгляда
        пользователь может наблюдать сцену с определенного ракурса. При изменении
        положения или направления камеры, обзор на сцену изменяется.
\end{itemize}

\section{Анализ способов задания моделей}

В компьютерной графике способов задания моделей три: каркасная, поверхностная и
объемная~\cite{prilipenko:2005}.

%\subsection{Каркасная модель}

Каркасная модель представляется набором вершин и ребер, соединяющих вершины.
Данное представление не всегда точно передает информацию об объектах сцены,
так как может неоднозначно определять его форму.

%\subsection{Поверхностная модель}

Поверхностная модель --- это модель, которая каким-либо образом задана поверхностями.
Описание прверхности может быть аналитическим или с помощью полигонов.

%\subsection{Объемная (твердотельная) модель}

Твердотельная форма задания модели отличается от поверхностной тем, что в таких моделях
задается информация о материале объекта.

\subsection{Выбор способа задания модели}

Для решения поставленной задачи необходимо правильное восприятие форм объекта, поэтому
каркасная модель не подходит для решения поставленной задачи. Твердотельная модель также не
подходит для решения задачи, так как неважно из чего состоят объекты сцены. Поверхностная
модель точно определяет форму объекта сцены, поэтому в данной программе можноиспользовать
поверхностную модель.


\section{Анализ алгоритмов удаления невидимых ребер и граней}

В компьюетрной графике важнейшей задачей является удаление невидимых ребер и граней.
Данные алгоритмы определяют, какие ребра и грани видны наблюдателю, находящегося
в определенной точке пространства относительно этих ребер и граней.

Алгоритмы делятся на способы решения. Первая группа алгоритмов решают задачу
в объектом пространстве, другие --- задачу в пространстве изображения.
Алгоритмы, работающие с объектным пространством,  дают более точные результаты,
однако медленнее алгоритмов, работающих в пространстве изображения.

\subsection{Алгоритм Робертса}

Данный алгоритм работает в объектном пространтсве. Решает задачу только с выпуклыми телами.
Алгоритм выполняется в 4 этапа~\cite{roberts:1964}:
\begin{itemize}
    \item подготовка исходных данных --- разбиение невыпуклых объектов на выпуклые, составление
        матрицы тела каждого из объектов;
    \item удаление ребер, экранируемых самим телом;
    \item удаление ребер, экранируемых другими телами;
    \item удаление линий пересечения тел, экранируемых
        самими телами и другими телами, связанными отношением протыкания.
\end{itemize}



\textbf{Преимущества}:
\begin{itemize}
    \item высокая точность определения видимости граней без приближений.
\end{itemize}


\textbf{Недостатки}:
\begin{itemize}
    \item Тела сцены должны быть выпуклыми, что не всегда так.
    \item Высокая вычислительная сложность при большом количестве граней. Рост
        вычислительной сложности --- квадрат количества граней.
\end{itemize}

\subsection{Z-буфер}

Данный подход работает в пространстве изображения.

Алгоритм Z-буфер~\cite{zbuffer:1974} использует два буфера: буфер глубины и буфер кадра.

Буфер глубины определяет для каждого пикселя информацию о глубине ближайшего к наблюдателю
объекта.  В буфере кадра хранится цвет пикселей.

\medskip

Алгоритм начинает работу с инициализации буфера глубины максимальным значением глубины.
При растеризации (отрисовке) каждого полигона вычисляется глубина каждого из его пикселей.
Если глубина очередного пикселя меньше соответствующей глубины в буфере глубины, то
изменяется соответствующая информация в буфере кадра и значение глубины этого пикселя
обновляется.

\textbf{Преимущества}:
\begin{itemize}
    \item возможность отрисовки объектов любой сложности;
    \item Линейная зависимость от количества пикселей, отрисовываемых на изображении;
    \item простота реализации.
\end{itemize}

\textbf{Недостатки}:
\begin{itemize}
    \item необходима дополнительная память для хранения буфера глубины;
    \item возможны артефакты при недостаточной точности буфера.
\end{itemize}


\subsection{Алгоритм обратной трассировки лучей}

Наблюдатель видит объект посредством испускаемого источником света,
который падает на этот объект и согласно законам оптики некоторым путем доходит
до глаза наблюдателя. Отслеживать пути лучей от источника
к наблюдателю неэффективно с точки зрения вычислений, поэтому наилучшим
способом будет отслеживание путей в обратном направлении, то есть от
наблюдателя к объекту.

Предполагается, что сцена уже преобразована в пространство изображения,
а точка, в которой находится наблюдатель, находится в бесконечности
на положительной полуоси 𝑧, и поэтому световые лучи параллельны этой же
оси. При этом каждый луч проходит через центр пикселя растра до сцены.
Траектория каждого луча отслеживается для определения факта пересечения
определенных объектов сцены с этими лучами. При этом необходимо
проверить пересечение каждого объекта сцены с каждым лучом, а пересечение
с $z_{min}$ представляет видимую поверхность для данного пикселя.
Если же точка наблюдателя находится не в бесконечности, то есть в рас-
смотрении фигурирует перспективная проекция, то предполагается, что сам
наблюдатель по-прежнему находится на положительной полуоси 𝑧, а сам
растр при этом перпендикуляром оси 𝑧. Задача будет состоять в том, чтобы
построить одноточечную центральную проекцию на картинную плоскость.


Данный алгоритм позволяет создавать фотореалистичные изображения с отражениями
и преломлениями лучей света.


\textbf{Преимущества}:
\begin{itemize}
    \item возможность отрисовки объектов любой сложности;
    \item точное моделирование оптических свойств световых лучей.
\end{itemize}

\textbf{Недостатки}:
\begin{itemize}
    \item высокая вычислительная сложность.
\end{itemize}


\subsection{Выбор алгоритма удаления невидимых ребер и граней}

Так как не требуется высокая реалистичность от объектов сцены, а форма ее объектов
может быть как выпуклой, так и невыпуклой, алгоритм Робертса не подходит для решения задачи.
Алгоритм обратной трассировки лучей также не подходит, потому что не требуется
фотореалистичные изображения с отражениями и преломлениями лучей света. Поэтому
для решения поставленной задачи был выбран алгоритм Z-буфер, потому что
в достаточной мере позволяет определить видимость граней и ребер и имеет
приемлемую скорость вычислений, в осномном не зависящую от количества объектов сцены.


\section{Анализ методов закраски}
Закраска определяет визуальную составляющую объектов сцены и влияет на их восприятие.

\subsection{Плоская закраска}

Плоская закраска подразумевает, что источник света находится в бесконечности. По этой причине
считается, что угол падения лучей света на поверхность одинаков. Так как интенсивность зависит от угла падения,
то это значит, что вся грань будет закрашена одним цветом.


\includeimage
{flat}
{f}
{h}
{0.4\textwidth}
{Плоская закраска}

\textbf{Преимущества}:
\begin{itemize}
    \item быстрая скорость обработки;
    \item простота реализации.
\end{itemize}

\textbf{Недостатки}:
\begin{itemize}
    \item возможны образования ребер между гранями, которых на самом деле нет;
    \item не учитывается плавный переход интенсивности освещения от источника света,
        находящегося на конечном расстоянии.
\end{itemize}


\subsection{Закраска по Гуро}

Закраска по Гуро~\cite{guro:1971} интерполирует интенсивность освещения между вершинами одной грани.
Интенсивность в вершинах вычисляется по нормалям в этих вершинах.
Интерполяция происходит в два этапа: интерполирование по ребрам и интерполирование
между ребрами.

Данная закраска хорошо подходит для диффузиозного отражения.

\includeimage
{guro}
{f}
{h}
{0.4\textwidth}
{Закраска по Гуро}


\textbf{Преимущества}:
\begin{itemize}
    \item градиент интенсивности освещения между вершинами грани;
\end{itemize}

\textbf{Недостатки}:
\begin{itemize}
    \item возможна потеря бликов света, так как они могут не попадать на вершины граней.
\end{itemize}


\subsection{Закраска по Фонгу}

\includeimage
{phong}
{f}
{h}
{0.4\textwidth}
{Закраска по Фонгу}

Закраска по Фонгу~\cite{phong:1998} работает аналогично закраске по Гуро, однако вместо интерполирование
интенсивности света на вершинах граней, интерполируются нормали этих граней. Поэтому
для рассчета интенсивности освещения потребуется в каждом пикселе считать интенсивность
относительно интерполированной нормали.

Данная закраска хорошо подходит для зеркального отражения.



\textbf{Преимущества}:
\begin{itemize}
    \item высокая реалистичность;
    \item блики света;
    \item плавные переходы света и тени.
\end{itemize}

\textbf{Недостатки}:
\begin{itemize}
    \item более высокая вычислительная сложность относительно других алгоритмов;
\end{itemize}


\subsection{Выбор алгоритма закраски}

Так как не требуется высокая реалистичность и модели представлены в виде граней с вершинами,
которые не имеют собственных нормалей, закраска по Фонгу не подходит для решения задачи.
Так как объекты сцены будут достаточно простыми, можно сделать выбор в пользу более
производительного алгоритма закраски. Поэтому выбираем алгоритм простой закраски, так
как работает быстрее остальных и в достаточной мере обеспечивает визуализацию объектов
сцены.



\section{Вывод}

В разделе были проанализированы существующие алгоритмы построения трехмерной сцены и выбраны
методы для решения поставленной задачи. Был выбран алгоритм Z-буфер с плоской закраской.
