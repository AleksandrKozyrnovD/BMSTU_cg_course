\chapter{Технологический раздел}

В данном разделе рассматривается выбор средств реализации, описывается структура
классов программы и приводится интерфейс программного обеспечения.

\section{Cредства реализации}

Для реализации программного обеспечения был выбран язык C++~\cite{cpp}.
Это обусловлено следующими причинами:
\begin{itemize}
    \item C++ обладает высокой вычислительной производительностью;
    \item обладает большим количеством литературы и примеров;
\end{itemize}


При написании программного обеспечения были задействованы библиотеки
SDL2~\cite{sdl}, GLM~\cite{glm} и Dear ImGui~\cite{imgui}. Библиотека SDL2 представляет из себя библиотеку,
представляющую из себя окно и дающая исполнителю функции для попиксельной
манипуляции с изображением. GLM библоитека используется по причине
наличия математических векторов и функций, полезных для компьютерной графики.
Dear ImGui библиотека используется для реализации пользовательского интерфейса.


\section{Структура классов программного обеспечения}

В данной диаграмме на рисунке (3.1) показаны основные
классы программного обеспечения.


\includeimage
{uml}
{f}
{h}
{0.85\textwidth}
{Структура основных классов программного обеспечения}


Описание некоторых классов, которые решают задачу программного обеспечения:
\begin{itemize}
    \item Model --- модель сцены, содержащая некоторое описание поверхностной модели объекта;
    \item SurfaceModel --- некоторая реализация поверхностной модели объекта. Состоит из
        массива граней Facet.
    \item CompositeObject --- класс, позволяющий группировать объекты в группы.
    \item Scene --- класс, представляющий из себя набор массив объектов и методов для
        добавления и удаления объектов.
    \item Camera --- класс, представляющий из себя точку наблюдения.
    \item Light --- класс, представляющий из себя точечный направленный источник света.
    \item ZBufMapper --- класс, представляющий из себя модифицированный алгоритм Z буффером.
    \item ShadowMapper --- класс, представляющий из себя алгоритм заполнения теневого буфера.
\end{itemize}


\section{Реализация заполнения теневого буфера}

В листинге \ref{lst:zbuf} представлен алгоритм заполнения теневого буффера
источника света.

\newpage

\begin{center}
\captionsetup{justification=raggedright,singlelinecheck=off}
\lstinputlisting[firstline=12, lastline=119, caption=Реализация заполнения теневого
буфера, label=lst:zbuf
]{/home/aleksandr/Desktop/bmstu/Curse/CGNEW/inc/Drawer/normalzbuffer/NewDrawVisitor.hpp}
\end{center}

\section{Интерфейс программного обеспечения}

На рисунках (3.4) --- (3.11) представлены элементы интерфейса программного обеспечения.

\includeimage
{int1}
{f}
{h}
{0.9\textwidth}
{Интерфейс программного обеспечения в виде панели задач}


\includeimage
{intfull}
{f}
{h}
{0.75\textwidth}
{Весь интерфейс программного обеспечения, включая выводимое изображение}

\includeimage
{inthelp}
{f}
{h}
{0.6\textwidth}
{Интерфейс окна с помощью}


На рисунках (3.7) --- (3.8) зеленые ячейки означают, что в данное место
можно поставить объект сцены, когда синие --- нельзя.
Количество ячеек зависит от заданного размера полотна сцены.
\includeimage
{intedit}
{f}
{h}
{0.6\textwidth}
{Интерфейс окна редактирования}

На рисунке (3.8) показано всего 3 опции в каждой ячейке:
создание нового объекта, удаление объекта (если имеется)
и изменение его положения на полотне.
\includeimage
{inteditopt}
{f}
{h}
{0.6\textwidth}
{Интерфейс окна редактирования с опциями}


На рисунке (3.9) предоставляется список из заранее спроектированных объектов сцены, которые
можно поставить на полотно сцены.
\includeimage
{inteditbrowser}
{f}
{h}
{0.6\textwidth}
{Интерфейс окна выбора моделей}

\includeimage
{intscene}
{f}
{h}
{0.6\textwidth}
{Интерфейс опций работы со сценой}



\includeimage
{intscenenew}
{f}
{h}
{0.6\textwidth}
{Интерфейс окна создания новой сцены}

\includeimage
{intsceneopen}
{f}
{h}
{0.6\textwidth}
{Интерфейс окна открытия ранее созданной сцены}


\includeimage
{intscene1}
{f}
{h}
{0.6\textwidth}
{Выводимое изображение при одинаковом направлении и расположении источника света и наблюдателя}

\includeimage
{intscene2}
{f}
{h}
{0.6\textwidth}
{Выводимое изображение при измененном положении наблюдателя}

\includeimage
{intscene3}
{f}
{h}
{0.6\textwidth}
{Выводимое изображение при измененном положении источника света}

\clearpage

\section{Вывод}
В данном разделе был рассмотрен выбор средств реализации, описана структура
классов программы и приведен интерфейс программного обеспечения.
