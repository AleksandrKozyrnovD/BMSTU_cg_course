\chapter{Конструкторский раздел}

В данном разделе представлены алгоритмы, выбранные для решения задачи,
рассмотрены структуры данных.

\section{Требования к программному обеспечению}

Программа должна выполнять следующую функции:
\begin{itemize}
    \item создание сцены с заданным размером полотна;
    \item перемещение и поворот камеры;
    \item перемещение источника света;
    \item добавление и удаление объектов сцены;
    \item перемещение объектов сцены.
\end{itemize}

\section{Общий алгоритм построения сцены}

На рисунках (2.1) --- (2.4) представлена IDEF0 диаграмма алгоритма построения
сцены.

\includeimage
{idef3}
{f}
{h}
{0.85\textwidth}
{Схема алгоритма уровня A0}


\includeimage
{idef2}
{f}
{h}
{0.85\textwidth}
{Декомпозиция уровня A0}

\includeimage
{idef1}
{f}
{h}
{0.85\textwidth}
{Декомпозиция уровня A2}

\includeimage
{idef0}
{f}
{h}
{0.85\textwidth}
{Декомпозиция уровня A3}

\clearpage

\section{Модифицированный алгоритм Z-буфер}

Для каждого источника света нужно инициализировать теневой Z-буфер. Далее
определить глубину для каждого пикселя теневого буфера источника света для
точки наблюдения из источника света.

После заполнения всех теневых буферов, следует выполнять основной алгоритм Z-буфера
относительно положения наблюдателя. При этом, если очередной пиксель виден, нужно проверить,
видим ли этот пиксель относительно какого-либо источника света.

Для определения видимости точки из буфера кадра, нужно рассматриваемую точку
$(x,y,z)$ преобразовать из системы координат наблюдателя в систему координат источника света $(x',y',z')$.
Если в теневом буфере $z'(x',y') < z_{shadow}(x',y')$, то
точка видна. Иначе она находится в тени.

На рисунке (2.1) показан модифицированный алгоритм Z-буфера.

\includeimage
{zbuf}
{f}
{h}
{0.75\textwidth}
{Модифицированный алгоритм Z-буфера}

\clearpage

\section{Используемые структуры данных и классы}

Для реализации работы программы используются следующие структуры данных:
\begin{enumerate}
    \item Сцена
        \begin{itemize}
            \item Массив объектов сцены;
            \item Камера и источник света;
            \item Методы добавления и удаления объектов сцены.
        \end{itemize}
    \item Составная трехмерная модель
        \begin{itemize}
            \item Массив объектов сцены, из которых состоит объект.
        \end{itemize}
    \item Объект сцены
        \begin{itemize}
            \item Матрица преобразования;
            \item Массив граней.
        \end{itemize}
    \item Грань
        \begin{itemize}
            \item 3 вершины;
            \item Нормаль.
            \item Цвет;
        \end{itemize}
    \item Вершина
        \begin{itemize}
            \item Координаты в пространстве;
        \end{itemize}
    \item Камера
        \begin{itemize}
            \item Координаты в пространстве;
            \item 3 вектора, определяющих направление камеры.
        \end{itemize}
    \item Источник света
        \begin{itemize}
            \item Координаты в пространстве;
            \item 3 вектора, определяющих направление источника света;
            \item Интенсивность света.
        \end{itemize}
\end{enumerate}


\section{Вывод}

В данном разделе были представлены алгоритмы, выбранные для решения задачи,
были рассмотрены структуры данных.
